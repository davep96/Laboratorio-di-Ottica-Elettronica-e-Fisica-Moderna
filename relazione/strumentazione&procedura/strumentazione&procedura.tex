% !TeX spellcheck = it_IT
\documentclass[a4paper,11pt]{article}

\usepackage[italian]{babel}

\usepackage[table,xcdraw]{xcolor}

\usepackage[latin1]{inputenc}

\usepackage[T1]{fontenc}

\usepackage{graphicx}

\usepackage{indentfirst}

\usepackage{amsmath,amssymb}

\usepackage{enumitem} 

\newcommand{\virgolette}[1]{``#1''}

\usepackage[margin=1in]{geometry} %Smaller margins

\usepackage{lmodern} %Vector PDF

\usepackage{siunitx}

\usepackage{xcolor}

\usepackage{colortbl}

\usepackage{multirow}

\usepackage{rotating}

\usepackage{booktabs}

\usepackage{longtable}

\usepackage{graphicx}

\usepackage{wrapfig}

\begin{document}
	\section{Strumentazione}
	
		Qui di seguito sono elencati i pezzi dell'apparato sperimentale usato per portare a termine l'esperimento.
		
		\begin{description}[align=left]
			
			\item [Generatore] necessario per fornire la differenza di potenziale che genera la forza operante sugli elettroni. Il generatore permette di applicare un voltaggio fino a $\SI{500}{\volt}$.
			
			\item [Commutatore] che permette di invertire la polarit� degli elettrodi, invertendo cos� anche il verso della forza.
		
			\item [Lampada] opportunamente regolabile in direzione e intensit� luminosa, che permette di illuminare le gocce d'olio da visualizzare.
			
			\item [Bolla d'aria] incastonata nel supporto dell'esperimento, che permette la messa in bolla dell'intero apparato.
			
			\item [Camera di Millikan] ossia il pezzo pi� importante dell'esperimento, luogo in cui avviene la caduta delle gocce d'olio. La camera � a sua volta composta da
			
			\begin{description}[align=left]
				
				\item[Una sede] provvista di delle guide su cui montare l'intera camera.
				
				\item[Un distanziatore] isolante e trasparente, attraverso cui � possibile osservare la caduta delle gocce. La lampada sopra citata illumina proprio in questo punto.
				
				\item[Due elettrodi] i cui ruoli di anodo e catodo sono decisi nel corso dell'esperimento tramite il commutatore. Uno si trova sotto la sede, l'altro sopra il distanziatore. A lore � applicata la differenza di potenziale del generatore; sono pertanto essenziali per esercitare la forza sugli elettroni.
				
				\item[Una parete laterale] che circonda l'intera camera in modo da tenere insieme e isolare dall'ambiente circostante tutti i pezzi. La parete � provvista di una lente attraverso cui guardare le gocce.
				
				\item[Un coperchio di chiusura] provvisto di un foro dove inserire lo spruzzatore delle goccioline d'olio.
				
			\end{description}
		
			\item [Ago] da inserire nella camera per la messa al fuoco delle goccioline. L'ago si trova infatti nella posizione in cui pi� o meno cadranno le goccioline.
			
			\item [Microscopio] dotato di un sistema di lenti per la messa a fuoco e di un reticolo da utilizzare per le misure della posizione delle goccioline.
			
			\item [Sorgente di Th] necessaria per una migliore ionizzazione delle gocce d'olio. Poich� le gocce uscivano gi� ionizzate dallo spruzzatore se ne poteva in realt� fare a meno.
			
			\item [Termistore] utilizzato per controllare la temperatura nella camera di Millikan.
			
			\item [Tester] connesso ai terminali del termistore per misurarne la resistenza elettrica. Sul supporto dell'esperimento era presente una tabella che permetteva di convertire il valore della resistenza nella temperatura d'esercizio.
			
			\item [Spruzzatore] d'olio per spruzzare le gocce all'interno della camera.
			
			\item [Un cellulare] di modello LG G4 usato come cronometro, in quanto pi� sensibile e facilmente leggibile di quello in dotazione. Era possibile utilizzare i tasti del volume per fermare e segnare un giro del cronometro.
			
		\end{description}
	
		Chiaramente � presente un supporto per tenere insieme tutti i pezzi citati.
		
	\section{Procedura sperimentale}
	Segue ora una descrizione della procedura usata per l'esperimento. Per prima cosa si sono fatte con un calibro sei misure in punti diversi del distanziatore isolante per poi determinare l'intensit� del campo elettrico.
	Si � poi montata la camera di Millikan e si � messo a fuoco l'ago inserito nel foro dell'elettrodo superiore attraverso cui sarebbero poi passate le gocce d'olio. Si � messo a fuoco anche il reticolo.
	Si � connesso il tester ai terminali del termistore e si � misurata la temperatura, cosa che � stata ripetuta all'inizio di ogni misura.
	Si sono poi prese le misure: si spruzzava inizialmente le gocce d'olio nella camera, con la sorgente di Th accesa per una decina di secondi. Una volta individuata poi una gocciolina d'olio visibile si verificava che la sua velocit� di caduta senza differenza di potenziale, quella di risalita e quella di caduta con la differenza di potenziale consentissero delle misure precise (se infatti una delle tre velocit�, tipicamente quella di caduta con differenza di potenziale, fosse stata troppo alta i nostri tempi di reazione avrebbero influito troppo sulla misura).
	Allo stesso tempo la velocit� di caduta libera non doveva essere troppo lenta poich� per esperienze, le gocce troppo leggere erano molto pi� soggette a perturbazioni esterne e spesso subivano rallentamenti o accelerazioni. Una volta individuata, non senza fatica, una goccia adatta all'osservazione si procedeva alla misurazione delle sue posizioni e dei relativi tempi.
	Si portava la goccia, tramite la forza elettrica, in cima al reticolo dopodich� si misurava il tempo che ci metteva a percorrere una tacca grande del reticolo (corrispondente a $\SI{0.5}{\milli \meter}$) per cinque tacche, si accendeva il generatore in maniera tale da far percorrere alla goccia lo stesso percorso in salita e la si faceva scendere nuovamente, questa volta applicando la differenza di potenziale.
	Si ottenevano in questo modo quindici misure. Ogni tanto si vedeva facilmente che la goccia accelerava o decelerava in presenza di corrente, sintomo di perdita di carica, per cui si scartavano i dati presi e si ricominciava con un'altra goccia.
	
	Si sono prese le misure pi� volte per diverse gocce e per differenze di potenziale di 200, 300 o 400 V. Al termine di ogni misura si trasferivano i tempi registrati col cronometro su un file excel e si verificava che la misura avesse senso.
	
	
\end{document}